\documentclass[]{article}
\usepackage{lmodern}
\usepackage{amssymb,amsmath}
\usepackage{ifxetex,ifluatex}
\usepackage{fixltx2e} % provides \textsubscript
\ifnum 0\ifxetex 1\fi\ifluatex 1\fi=0 % if pdftex
  \usepackage[T1]{fontenc}
  \usepackage[utf8]{inputenc}
\else % if luatex or xelatex
  \ifxetex
    \usepackage{mathspec}
  \else
    \usepackage{fontspec}
  \fi
  \defaultfontfeatures{Ligatures=TeX,Scale=MatchLowercase}
\fi
% use upquote if available, for straight quotes in verbatim environments
\IfFileExists{upquote.sty}{\usepackage{upquote}}{}
% use microtype if available
\IfFileExists{microtype.sty}{%
\usepackage{microtype}
\UseMicrotypeSet[protrusion]{basicmath} % disable protrusion for tt fonts
}{}
\usepackage[margin=1in]{geometry}
\usepackage{hyperref}
\hypersetup{unicode=true,
            pdftitle={Homework 1},
            pdfborder={0 0 0},
            breaklinks=true}
\urlstyle{same}  % don't use monospace font for urls
\usepackage{graphicx,grffile}
\makeatletter
\def\maxwidth{\ifdim\Gin@nat@width>\linewidth\linewidth\else\Gin@nat@width\fi}
\def\maxheight{\ifdim\Gin@nat@height>\textheight\textheight\else\Gin@nat@height\fi}
\makeatother
% Scale images if necessary, so that they will not overflow the page
% margins by default, and it is still possible to overwrite the defaults
% using explicit options in \includegraphics[width, height, ...]{}
\setkeys{Gin}{width=\maxwidth,height=\maxheight,keepaspectratio}
\IfFileExists{parskip.sty}{%
\usepackage{parskip}
}{% else
\setlength{\parindent}{0pt}
\setlength{\parskip}{6pt plus 2pt minus 1pt}
}
\setlength{\emergencystretch}{3em}  % prevent overfull lines
\providecommand{\tightlist}{%
  \setlength{\itemsep}{0pt}\setlength{\parskip}{0pt}}
\setcounter{secnumdepth}{5}
% Redefines (sub)paragraphs to behave more like sections
\ifx\paragraph\undefined\else
\let\oldparagraph\paragraph
\renewcommand{\paragraph}[1]{\oldparagraph{#1}\mbox{}}
\fi
\ifx\subparagraph\undefined\else
\let\oldsubparagraph\subparagraph
\renewcommand{\subparagraph}[1]{\oldsubparagraph{#1}\mbox{}}
\fi

%%% Use protect on footnotes to avoid problems with footnotes in titles
\let\rmarkdownfootnote\footnote%
\def\footnote{\protect\rmarkdownfootnote}

%%% Change title format to be more compact
\usepackage{titling}

% Create subtitle command for use in maketitle
\newcommand{\subtitle}[1]{
  \posttitle{
    \begin{center}\large#1\end{center}
    }
}

\setlength{\droptitle}{-2em}

  \title{Homework 1}
    \pretitle{\vspace{\droptitle}\centering\huge}
  \posttitle{\par}
  \subtitle{Joseph Sepich}
  \author{}
    \preauthor{}\postauthor{}
    \date{}
    \predate{}\postdate{}
  

\begin{document}
\maketitle

{
\setcounter{tocdepth}{2}
\tableofcontents
}
\section{Problem 1}\label{problem-1}

\subsection{a.}\label{a.}

Functions as given:

\begin{enumerate}
\def\labelenumi{\arabic{enumi}.}
\tightlist
\item
  \(log^2(n)\)
\item
  \(n \choose 2\)
\item
  \(log(n^2)\)
\item
  \(log(n!)\)
\item
  \(2^{log(n)}\)
\item
  \(n*log(n)\)
\item
  \(4^{log(n)}\)
\item
  \(\sqrt n\)
\item
  \(2^{log^2(n)}\)
\item
  \(n\)
\item
  \(log(log(n))\)
\end{enumerate}

Recall the definition of big O notation:

c \textgreater{} 0, n\textsubscript{0} \textgreater{} 0 and
\(0 \leq f(n) \leq cg(n)\) for all \(n \geq n_0\)

We will proceed by comparing each function to the previous and adjusting
the order accordingly.

Recall that \(n \choose 2\)= \(\frac{n!}{2(n-2)!}\)

With n\textsubscript{0} = 1 and c = 1

\[log^2(n) \leq \binom{n}{2} \]

So we start with:

\begin{enumerate}
\def\labelenumi{\arabic{enumi}.}
\tightlist
\item
  \(log^2(n)\)
\item
  \(n \choose 2\)
\end{enumerate}

Now compare function 2. and 3.

Here we can choose n\textsubscript{0} = 2 and c = 2

\[log(n^2) \leq \frac{2n!}{2(n-2)!} = n(n-1)\]

So compare 3. and 1.

Here we can choose n\textsubscript{0} = 2 and c = 2

\[log(n^2) \leq 2log^2(n)\]

So we have:

\begin{enumerate}
\def\labelenumi{\arabic{enumi}.}
\tightlist
\item
  \(log(n^2)\)
\item
  \(log^2(n)\)
\item
  \(n \choose 2\)
\end{enumerate}

Now compare 3. and 4.

Here we can choose n\textsubscript{0} = 2 and c = 1

\[log(n!) \leq \binom{n}{2}\]

And compare to \(log^2(n)\)

We can choose n\textsubscript{0} = 1 and c = 1/2

\[log^2(n) \leq 2log(n!)\]

So we have:

\begin{enumerate}
\def\labelenumi{\arabic{enumi}.}
\tightlist
\item
  \(log(n^2)\)
\item
  \(log^2(n)\)
\item
  \(log(n!)\)
\item
  \(n \choose 2\)
\end{enumerate}

Next looking at \(2^{log(n)}\) we know equal x by logarithm properties,
so we can put it between the two functions that increase at an
increasing rate and increase at a decreasing rate and have:

\begin{enumerate}
\def\labelenumi{\arabic{enumi}.}
\tightlist
\item
  \(log(n^2)\)
\item
  \(log^2(n)\)
\item
  \(2^{log(n)}\)
\item
  \(log(n!)\)
\item
  \(n \choose 2\)
\end{enumerate}

Next we look at \(n*log(n)\)

This function increases at an increasing rate, so we know it is above
\(2^{log(n)}\) then let's compare to the next highest \(log(n!)\)

Choosing n\textsubscript{0} = 1 and c = 1 we get

\[n*log(n) \leq log(n!)\]

So we have:

\begin{enumerate}
\def\labelenumi{\arabic{enumi}.}
\tightlist
\item
  \(log(n^2)\)
\item
  \(log^2(n)\)
\item
  \(2^{log(n)}\)
\item
  \(n*log(n)\)
\item
  \(log(n!)\)
\item
  \(n \choose 2\)
\end{enumerate}

Next we look at \(4^{log(n)}\). This is similar to the exponential
function \(4^n\), so compare it to our fastest growing function:

Choosing n\textsubscript{0} = 1 and c = 1 we get

\[\binom{n}{2} \leq 4^{log(n)}\]

So we have:

\begin{enumerate}
\def\labelenumi{\arabic{enumi}.}
\tightlist
\item
  \(log(n^2)\)
\item
  \(log^2(n)\)
\item
  \(2^{log(n)}\)
\item
  \(n*log(n)\)
\item
  \(log(n!)\)
\item
  \(n \choose 2\)
\item
  \(4^{log(n)}\)
\end{enumerate}

Next we look at \(\sqrt n\), which obviously grows less than the linear
function \(2^{log(n)}\). Let's compare to our bottom two functions.

Choose n\textsubscript{0} = 2 and c = 2

\[\sqrt n \leq 2log^2(n)\]

and choose n\textsubscript{0} = 1 and c = 3

\[log(n^2) \leq 3\sqrt n\]

So we have:

\begin{enumerate}
\def\labelenumi{\arabic{enumi}.}
\tightlist
\item
  \(log(n^2)\)
\item
  \(\sqrt n\)
\item
  \(log^2(n)\)
\item
  \(2^{log(n)}\)
\item
  \(n*log(n)\)
\item
  \(log(n!)\)
\item
  \(n \choose 2\)
\item
  \(4^{log(n)}\)
\end{enumerate}

log(log(n)) is the slowest growing with the composed log functions and n
is the same function as \(2^{log(n)}\).

Keeping with this process I get:

\begin{enumerate}
\def\labelenumi{\arabic{enumi}.}
\tightlist
\item
  \(log(log(n)\)
\item
  \(log(n^2)\)
\item
  \(log^2(n)\)
\item
  \(\sqrt n\)
\item
  \(2^{log(n)}\)
\item
  \(n\)
\item
  \(n*log(n)\)
\item
  \(log(n!)\)
\item
  \(n \choose 2\)
\item
  \(4^{log(n)}\)
\item
  \(2^{log^2(n)}\)
\end{enumerate}

\subsection{b.}\label{b.}

Now we want to sort the functions into classes that are upper and lower
bounded by each other. Basically can you find an n\textsubscript{0} and
c for each other that it is upper bounded and a different one that they
are lower bounded.

There are only two groups of classes that I can see functions belonging
to. The others are not grouped with anyone else.

\begin{enumerate}
\def\labelenumi{\arabic{enumi}.}
\tightlist
\item
  \(n\) and \(2^{log(n)}\) grow at the exact same rate \(\Theta n\).
\item
  \(\binom{n}{2}\) and \(4^{log(n)}\) are both bounded by
  \(\Theta(n^2)\)
\end{enumerate}

The second group makes sense, because n choose 2 is the binomial
coefficient, or the coefficient for the term x\^{}2, so it grows at a
rate bounded \(\Theta(x^2)\). \(4^{log(n)}\) is equivalent to
x\textsuperscript{2}, so it must be bounded.

\section{Problem 2}\label{problem-2}

We are running the insertion sort algorithm on the following input
sequence:

2,1,4,3,6,5,\ldots{},n,n-1

We can see that each pair of numbers is switch around in order i.e.~1 is
where 2 should be and 2 is where 1 should be, 3 is where 4 should be and
4 is where 3 should be and so on up to n is where n-1 should be and n-1
is where n should be.

This means that for every 2 numbers we need to perform 4 actions (with
exception of the first pair, since we start at the second index). This
means we will perform \(4 * \frac{n}2\) actions or 2n actions. 2n is
bother lower and upper bounded by n, if you use n\textsubscript{0} = 1,
c = 1, and n\textsubscript{0} = 1, c = 3 respectively. Therefore the
running time of the input here is \(\Theta(n)\).


\end{document}
