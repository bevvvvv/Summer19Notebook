\documentclass[]{article}
\usepackage{lmodern}
\usepackage{amssymb,amsmath}
\usepackage{ifxetex,ifluatex}
\usepackage{fixltx2e} % provides \textsubscript
\ifnum 0\ifxetex 1\fi\ifluatex 1\fi=0 % if pdftex
  \usepackage[T1]{fontenc}
  \usepackage[utf8]{inputenc}
\else % if luatex or xelatex
  \ifxetex
    \usepackage{mathspec}
  \else
    \usepackage{fontspec}
  \fi
  \defaultfontfeatures{Ligatures=TeX,Scale=MatchLowercase}
\fi
% use upquote if available, for straight quotes in verbatim environments
\IfFileExists{upquote.sty}{\usepackage{upquote}}{}
% use microtype if available
\IfFileExists{microtype.sty}{%
\usepackage{microtype}
\UseMicrotypeSet[protrusion]{basicmath} % disable protrusion for tt fonts
}{}
\usepackage[margin=1in]{geometry}
\usepackage{hyperref}
\hypersetup{unicode=true,
            pdftitle={Chapter 9: Numerical Methods for ODE},
            pdfborder={0 0 0},
            breaklinks=true}
\urlstyle{same}  % don't use monospace font for urls
\usepackage{graphicx,grffile}
\makeatletter
\def\maxwidth{\ifdim\Gin@nat@width>\linewidth\linewidth\else\Gin@nat@width\fi}
\def\maxheight{\ifdim\Gin@nat@height>\textheight\textheight\else\Gin@nat@height\fi}
\makeatother
% Scale images if necessary, so that they will not overflow the page
% margins by default, and it is still possible to overwrite the defaults
% using explicit options in \includegraphics[width, height, ...]{}
\setkeys{Gin}{width=\maxwidth,height=\maxheight,keepaspectratio}
\IfFileExists{parskip.sty}{%
\usepackage{parskip}
}{% else
\setlength{\parindent}{0pt}
\setlength{\parskip}{6pt plus 2pt minus 1pt}
}
\setlength{\emergencystretch}{3em}  % prevent overfull lines
\providecommand{\tightlist}{%
  \setlength{\itemsep}{0pt}\setlength{\parskip}{0pt}}
\setcounter{secnumdepth}{5}
% Redefines (sub)paragraphs to behave more like sections
\ifx\paragraph\undefined\else
\let\oldparagraph\paragraph
\renewcommand{\paragraph}[1]{\oldparagraph{#1}\mbox{}}
\fi
\ifx\subparagraph\undefined\else
\let\oldsubparagraph\subparagraph
\renewcommand{\subparagraph}[1]{\oldsubparagraph{#1}\mbox{}}
\fi

%%% Use protect on footnotes to avoid problems with footnotes in titles
\let\rmarkdownfootnote\footnote%
\def\footnote{\protect\rmarkdownfootnote}

%%% Change title format to be more compact
\usepackage{titling}

% Create subtitle command for use in maketitle
\newcommand{\subtitle}[1]{
  \posttitle{
    \begin{center}\large#1\end{center}
    }
}

\setlength{\droptitle}{-2em}

  \title{Chapter 9: Numerical Methods for ODE}
    \pretitle{\vspace{\droptitle}\centering\huge}
  \posttitle{\par}
  \subtitle{Joseph Sepich}
  \author{}
    \preauthor{}\postauthor{}
    \date{}
    \predate{}\postdate{}
  

\begin{document}
\maketitle

{
\setcounter{tocdepth}{2}
\tableofcontents
}
\section{Problem 1 Scalar ODE}\label{problem-1-scalar-ode}

\[x' = 2x^2 + x - 1, x(1) = 1\]

\subsection{a.}\label{a.}

Write out Euler's method for this ODE. Compute the value of x(1.2) usign
Euler's method, with h = 0.1.

Recall Euler's method itself:

\[x_1 = x_0 + hx'(t_0) = x_0 + h*f(t_0,x_0)\]

Let's compute to x(1.2):

\[x(1.1) = x_0 + hf(t_0,x_0) = 1 + 0.1 * 2 = 1.2\]
\[x(1.2) = x_1 + hf(t_1,x_1) = 1.2 + 0.1 * 3.08 = 1.508\]

\subsection{b.}\label{b.}

Write out Heun's method for this ODE. Compute the value x(1.2) by Heun's
method with h = 0.1.

Recall Heun's method:

\[x_{k+1} = x_k + \frac12(K_1 + K_2)\] \[K_1 = h*f(t_k, x_k)\]
\[K_2 = h*f(t_k + h, x_k + K_1)\]

Let's compute to x(1.2):

At \(x_0 = 1, t_0 = 1\):

\[K_1 = 0.1 * 2 = 0.2\] \[K_2 = 0.1 * 3.08 = 0.308\]
\[x(1.1) = 1 + \frac12(0.2 + 0.308) = 1.254\]

At \(x_1 = 1.254, t_0 = 1.1\):

\[K_1 = 0.1 * 3.399 = 0.3399\] \[K_2 = 0.1 * 5.6749 = 0.5675\]
\[x(1.2) = 1.254 + \frac12(0.3399 + 0.5675) = 1.7077\]

\subsection{c}\label{c}

Write out the classic 4th order RK method for this ODE. Compute the
value x(1.2) by RK4 method with h = 0.1.

Recall RK4 method:

\[x_{k+1} = x_k + \frac16(K_1 + 2K_2 + 2K_3 + K_4)\]
\[K_1 = h*f(t_k, x_k)\] \[K_2 = h*f(t_k + \frac12h, x_k + \frac12K_1)\]
\[K_3 = h*f(t_k + \frac12h, x_k + \frac12K_2)\]
\[K_4 = h*f(t_k + h, x_k + K_3)\]

At \(x_0 = 1, t_0 = 1\):

\[K_1 = 0.1 * 2 = 0.2\] \[K_2 = 0.1 * 2.52 = 0.252 \]
\[K_3 = 0.1 * 2.6618 = 0.2662\] \[K_4 = 0.1 * 3.2333 = 0.3233\]
\[x(1.1) = 1 + \frac16(0.2 + 2 * 0.252 + 2 * 0.2262 + 0.3233) = 1.247\]

At \(x_1 = 1.247, t_0 = 1.1\)

\[K_1 = 0.1 * 1.802 = 0.1802\] \[K_2 = 0.1 * 2.467 = 0.2467\]
\[K_3 = 0.1 * 2.647 = 0.2647\] \[K_4 = 0.1 * 3.464 = 0.3464\]
\[x(1.2) = 1.247 + \frac16(0.1802 + 2 * 0.2467 + 2 * 0.2647 + 0.3464) = 1.505\]

\subsection{d}\label{d}

Write out 2nd order ABM method for this ODE. Note that this is a
multi-step method. it needs 2 initial values to initiate the iterations.
For the second initial value you can use the result obtained in part b
with Heun's method. Compute the values x(1.2) and x(1.3) using the ABM
method.

Here we add the intial value \(x_1 = 1.1, t_1 = 1.254\). Recall the ABM
method:

\[x_{n+1} = x_n + \frac{h}2(3f(t_n,x_n) - f(t_{n-1}, x_{n-1}))\]

To compute \(x(1.2) = x_2\):

\[x(1.2) = 1.254 + 0.1/ 2 * (3*3.399 - 2) = 1.664\]

To compute \(x(1.3) = x_3\):

\[x(1.3) = 1.664 + 0.1/2 *(3*6.2018 - 3.399) = 2.424\]

\section{Problem 2}\label{problem-2}

Given the following high order ordinary differential equation

\[x'''(t) = -2x'' + xt, x(0) = 1, x'(0) = 2, x''(0) = 3\]

\subsection{a}\label{a}

Rewrite the equation into a system of first order equations. Make sure
to include the intial conditions.

\[
\begin{cases} 
      x'_1 = x_2 \\
      x'_2 = x_3 \\
      x'_3 = -2x_3 + x_1t\\  
\end{cases}
\]

Initial values at \(t = 0\):

\[
\begin{cases} 
      x_1 = 1 \\
      x_2 = 2 \\
      x_3 = 3 \\
\end{cases}
\]

\subsection{b}\label{b}

Set up the forward Euler step for the system in (a) with h = 0.1

Recall the forward Euler method:

\[x_{k+1} = x_k + hx'(t_k) = x_k + h*f(t_k,x_k)\]

Here we have the values:

\[x_k = (x_1^0, x_2^0, x_3^0)^T\]

This first vector is our initial condition.

\[x_k' = (x_1', x_2',x_3')^T = (x_2^0, x_3^0, -2x_3^0+x_1^0t_0)^T\]

This gives us the setup:

\[x_{k+1} = x_k + 0.1 * x_k' = (x_1^0, x_2^0, x_3^0)^T + 0.1 * (x_2^0, x_3^0, -2x_3^0+x_1^0t_0)^T\]

\subsection{c}\label{c-1}

Write a Matlab script that computes 10 Euler steps, to obtain the value
of \(x_10 \approx x(1)\)

My scipt:

\begin{verbatim}
% Setup intial conditions
% t starts at zero
t = 0;
h = 0.1;
xPrev = [1; 2; 3;];
xPrevPrime = [xPrev(2); xPrev(3); myFunc(xPrev, t)];

for i = 1:10
   xPrev =  xPrev + h .* xPrevPrime;
   xPrevPrime = [xPrev(2); xPrev(3); myFunc(xPrev, t)];
   % increment t
   t = t + h;
end

disp(xPrev)
disp(t)

% differential equations
function out=myFunc(x, t) 
    out = -2 * x(3) + x(1) * t;
end
\end{verbatim}

My output (value of x(1) and value of t):

Problem2Script

\begin{itemize}
\item
  3.8514
\item
  3.4875
\item
  1.0241
\item
  1.0000
\end{itemize}


\end{document}
