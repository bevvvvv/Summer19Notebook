\documentclass[]{article}
\usepackage{lmodern}
\usepackage{amssymb,amsmath}
\usepackage{ifxetex,ifluatex}
\usepackage{fixltx2e} % provides \textsubscript
\ifnum 0\ifxetex 1\fi\ifluatex 1\fi=0 % if pdftex
  \usepackage[T1]{fontenc}
  \usepackage[utf8]{inputenc}
\else % if luatex or xelatex
  \ifxetex
    \usepackage{mathspec}
  \else
    \usepackage{fontspec}
  \fi
  \defaultfontfeatures{Ligatures=TeX,Scale=MatchLowercase}
\fi
% use upquote if available, for straight quotes in verbatim environments
\IfFileExists{upquote.sty}{\usepackage{upquote}}{}
% use microtype if available
\IfFileExists{microtype.sty}{%
\usepackage{microtype}
\UseMicrotypeSet[protrusion]{basicmath} % disable protrusion for tt fonts
}{}
\usepackage[margin=1in]{geometry}
\usepackage{hyperref}
\hypersetup{unicode=true,
            pdftitle={Chapter 6: Linear Systems},
            pdfborder={0 0 0},
            breaklinks=true}
\urlstyle{same}  % don't use monospace font for urls
\usepackage{graphicx,grffile}
\makeatletter
\def\maxwidth{\ifdim\Gin@nat@width>\linewidth\linewidth\else\Gin@nat@width\fi}
\def\maxheight{\ifdim\Gin@nat@height>\textheight\textheight\else\Gin@nat@height\fi}
\makeatother
% Scale images if necessary, so that they will not overflow the page
% margins by default, and it is still possible to overwrite the defaults
% using explicit options in \includegraphics[width, height, ...]{}
\setkeys{Gin}{width=\maxwidth,height=\maxheight,keepaspectratio}
\IfFileExists{parskip.sty}{%
\usepackage{parskip}
}{% else
\setlength{\parindent}{0pt}
\setlength{\parskip}{6pt plus 2pt minus 1pt}
}
\setlength{\emergencystretch}{3em}  % prevent overfull lines
\providecommand{\tightlist}{%
  \setlength{\itemsep}{0pt}\setlength{\parskip}{0pt}}
\setcounter{secnumdepth}{5}
% Redefines (sub)paragraphs to behave more like sections
\ifx\paragraph\undefined\else
\let\oldparagraph\paragraph
\renewcommand{\paragraph}[1]{\oldparagraph{#1}\mbox{}}
\fi
\ifx\subparagraph\undefined\else
\let\oldsubparagraph\subparagraph
\renewcommand{\subparagraph}[1]{\oldsubparagraph{#1}\mbox{}}
\fi

%%% Use protect on footnotes to avoid problems with footnotes in titles
\let\rmarkdownfootnote\footnote%
\def\footnote{\protect\rmarkdownfootnote}

%%% Change title format to be more compact
\usepackage{titling}

% Create subtitle command for use in maketitle
\newcommand{\subtitle}[1]{
  \posttitle{
    \begin{center}\large#1\end{center}
    }
}

\setlength{\droptitle}{-2em}

  \title{Chapter 6: Linear Systems}
    \pretitle{\vspace{\droptitle}\centering\huge}
  \posttitle{\par}
  \subtitle{Joseph Sepich}
  \author{}
    \preauthor{}\postauthor{}
    \date{}
    \predate{}\postdate{}
  

\begin{document}
\maketitle

{
\setcounter{tocdepth}{2}
\tableofcontents
}
\section{Problem 1 Gaussian
Elimination}\label{problem-1-gaussian-elimination}

\subsection{a.}\label{a.}

Consider the following 3x3 system:

\[3.3330x_1 + 15920x_2 - 10.333x_3 = 7973.6\]
\[2.2220x_1 + 16.710x_2 + 9.6120x_3 = 0.96500\]
\[-1.5611x_1 + 5.1792x_2 - 1.6855x_3 = 2.7140\]

Verify that the solution is \(x_1 = 1, x_2 = 0.5, x_3 = -1\).

Solve the following system using naive Gaussian elimination with a
calculator. All calculations should use 5 significant digits. Show
details of your work.

Starting matrix:

\[
\left(\begin{array}{ccc} 
3.3330 & 15920 & -10.333\\
2.2220 & 16.710 & 9.6120\\
-1.5611 & 5.1792 & -1.6855
\end{array}\right)
\left(\begin{array}{c} 
x_1 \\
x_2 \\
x_3
\end{array}\right) =
\left(\begin{array}{c}
7973.6 \\
0.96500 \\
2.7140
\end{array}\right)
\]

Step 1: \(a_1 = \frac1{3.333} a_1\)

\[
\left(\begin{array}{ccc} 
1 & 477.65 & -3.1002\\
2.2220 & 16.710 & 9.6120\\
-1.5611 & 5.1792 & -1.6855
\end{array}\right)
\left(\begin{array}{c} 
x_1 \\
x_2 \\
x_3
\end{array}\right) =
\left(\begin{array}{c}
239.23 \\
0.96500 \\
2.7140
\end{array}\right)
\]

Step 2: \(a_2 = a_2 - 2.222a_1\)

\[
\left(\begin{array}{ccc} 
1 & 477.65 & -3.1002\\
0 & -1059.7 & 16.501\\
-1.5611 & 5.1792 & -1.6855
\end{array}\right)
\left(\begin{array}{c} 
x_1 \\
x_2 \\
x_3
\end{array}\right) =
\left(\begin{array}{c}
239.23 \\
-5314.8 \\
2.7140
\end{array}\right)
\]

Step 3: \(a_3 = a_3 + 1.5611a_1\)

\[
\left(\begin{array}{ccc} 
1 & 477.65 & -3.1002\\
0 & -1059.7 & 16.501\\
0 & 7461.7 & -6.5252
\end{array}\right)
\left(\begin{array}{c} 
x_1 \\
x_2 \\
x_3
\end{array}\right) =
\left(\begin{array}{c}
239.23 \\
-5314.8 \\
3737.4
\end{array}\right)
\]

Step 4: \(a_2 = \frac1{-1059.7}a_2\)

\[
\left(\begin{array}{ccc} 
1 & 477.65 & -3.1002\\
0 & 1 & -0.00156\\
0 & 7461.7 & -6.5252
\end{array}\right)
\left(\begin{array}{c} 
x_1 \\
x_2 \\
x_3
\end{array}\right) =
\left(\begin{array}{c}
239.23 \\
0.50155 \\
3737.4
\end{array}\right)
\]

Step 5: \(a_3 = a_3 - 7461.7 * a_2\)

\[
\left(\begin{array}{ccc} 
1 & 477.65 & -3.1002\\
0 & 1 & -0.00156\\
0 & 0 & 5.0939
\end{array}\right)
\left(\begin{array}{c} 
x_1 \\
x_2 \\
x_3
\end{array}\right) =
\left(\begin{array}{c}
239.23 \\
0.50155 \\
-5.0938
\end{array}\right)
\]

Step 6: \(a_3 = \frac{1}{5.0939}\)

\[
\left(\begin{array}{ccc} 
1 & 477.65 & -3.1002\\
0 & 1 & -0.00156\\
0 & 0 & 1
\end{array}\right)
\left(\begin{array}{c} 
x_1 \\
x_2 \\
x_3
\end{array}\right) =
\left(\begin{array}{c}
239.23 \\
0.50155 \\
-0.99999
\end{array}\right)
\]

Step 7: \(a_1 = a_1 - 477.65 a_2\)

\[
\left(\begin{array}{ccc} 
1 & 0 & -2.3551\\
0 & 1 & -0.00156\\
0 & 0 & 1
\end{array}\right)
\left(\begin{array}{c} 
x_1 \\
x_2 \\
x_3
\end{array}\right) =
\left(\begin{array}{c}
-0.33536 \\
0.50155 \\
-0.99999
\end{array}\right)
\]

Step 8: \(a_1 = a_1 + 2.3551a_3\)

\[
\left(\begin{array}{ccc} 
1 & 0 & 0\\
0 & 1 & -0.00156\\
0 & 0 & 1
\end{array}\right)
\left(\begin{array}{c} 
x_1 \\
x_2 \\
x_3
\end{array}\right) =
\left(\begin{array}{c}
0.99997 \\
0.50155 \\
-0.99999
\end{array}\right)
\]

Step 9: \(a_1 = a_1 + 0.00156a_3\)

\[
\left(\begin{array}{ccc} 
1 & 0 & 0\\
0 & 1 & 0\\
0 & 0 & 1
\end{array}\right)
\left(\begin{array}{c} 
x_1 \\
x_2 \\
x_3
\end{array}\right) =
\left(\begin{array}{c}
0.99997 \\
0.499996 \\
-0.99999
\end{array}\right)
\]

I get the answer as state above. Not the exact numbers due to rounding
errors.

\subsection{b.}\label{b.}

Consider the 2x2 system of linear equations.

\[1.13x + 1.54y = 4.21\] \[1.14x + 1.57y = 4.28\]

Verify that x = 1, y = 2 is the solution.

Solve using Gaussian to 3 significant digits.

\[
\left(\begin{array}{ccc} 
1.13 & 1.54 \\
1.14 & 1.57
\end{array}\right)
\left(\begin{array}{c} 
x \\
y
\end{array}\right) =
\left(\begin{array}{c}
4.21 \\
4.28
\end{array}\right)
\]

\[
\left(\begin{array}{ccc} 
1.13 & 1.54 \\
0 & 0.0164
\end{array}\right)
\left(\begin{array}{c} 
x \\
y
\end{array}\right) =
\left(\begin{array}{c}
4.21 \\
0.0327
\end{array}\right)
\]

\[
\left(\begin{array}{ccc} 
1.13 & 0 \\
0 & 0.0164
\end{array}\right)
\left(\begin{array}{c} 
x \\
y
\end{array}\right) =
\left(\begin{array}{c}
1.14 \\
0.0327
\end{array}\right)
\]

\[
\left(\begin{array}{ccc} 
1 & 0 \\
0 & 1
\end{array}\right)
\left(\begin{array}{c} 
x \\
y
\end{array}\right) =
\left(\begin{array}{c}
1.01 \\
1.99
\end{array}\right)
\]

The result of the elimination is the answer given above.

\section{Problem 2. Gauss in Matlab}\label{problem-2.-gauss-in-matlab}

When running the script I get a warning for the last vector. The
outputted results are similar to the previous ones when using pivoting,
but are very small when using naive Gaussian elimination. It is likely
that the naive elimination is creating a large loss of significance.

Script:

\begin{verbatim}

% Construct A and b in the following way

% Case 1
c = [0.2:0.2:1];
A = vander(c);
xsol = ones(size(c'));
b = A*xsol;
disp('Guassian:')
naiv_gauss(A,b)
disp('Pivoting:')
A\b

% Case 2
c = [0.1:0.1:1];
A = vander(c);
xsol = ones(size(c'));
b = A*xsol;
disp('Guassian:')
naiv_gauss(A,b)
disp('Pivoting:')
A\b


% Case 3
c = [0.05:0.05:1];
A = vander(c);
xsol = ones(size(c'));
b = A*xsol;
disp('Guassian:')
naiv_gauss(A,b)
disp('Pivoting:')
A\b
\end{verbatim}

Output:

Problem2Script Guassian:

ans =

\begin{itemize}
\tightlist
\item
  0.999999999999612
\item
  1.000000000000889
\item
  0.999999999999494
\item
  1.000000000000104
\item
  0.999999999999993
\end{itemize}

Pivoting:

ans =

\begin{itemize}
\tightlist
\item
  1.000000000000032
\item
  0.999999999999921
\item
  1.000000000000070
\item
  0.999999999999974
\item
  1.000000000000003
\end{itemize}

Guassian:

ans =

\begin{itemize}
\tightlist
\item
  0.998774840610394
\item
  1.005853289548497
\item
  0.988068244628706
\item
  1.013568750263305
\item
  0.990551321010233
\item
  1.004157217864660
\item
  0.998851916215852
\item
  1.000190204544961
\item
  0.999983061158011
\item
  1.000000606254654
\end{itemize}

Pivoting:

ans =

\begin{itemize}
\tightlist
\item
  1.000000000012049
\item
  0.999999999936912
\item
  1.000000000142096
\item
  0.999999999820241
\item
  1.000000000139617
\item
  0.999999999931755
\item
  1.000000000020688
\item
  0.999999999996307
\item
  1.000000000000348
\item
  0.999999999999987
\end{itemize}

Guassian:

ans =

1.0e+15 *

\begin{itemize}
\tightlist
\item
  0.100260367617011
\item
  -0.547611736692488
\item
  1.322086302655506
\item
  -1.861303854966565
\item
  1.697314314649383
\item
  -1.052455338465164
\item
  0.452613654373408
\item
  -0.134745815870952
\item
  0.027010370505822
\item
  -0.003364454050715
\item
  0.000193583684468
\item
  0.000007778825174
\item
  -0.000002060199317
\item
  0.000000130012968
\item
  -0.000000003878228
\item
  0.000000000134212
\item
  -0.000000000006845
\item
  0.000000000000244
\item
  -0.000000000000009
\item
  0.000000000000001
\end{itemize}

Pivoting:

Warning: Matrix is close to singular or badly scaled. Results may be
inaccurate. RCOND = 5.884995e-18.

In Problem2Script (line 33)

ans =

\begin{itemize}
\tightlist
\item
  1.001948529862383
\item
  0.980676737577058
\item
  1.089164834273452
\item
  0.745765429700010
\item
  1.501655183245481
\item
  0.272979675952538
\item
  1.801520667803391
\item
  0.313048153146008
\item
  1.463758890491371
\item
  0.751644013870925
\item
  1.105732817783706
\item
  0.964296230726580
\item
  1.009498367655909
\item
  0.998032829037259
\item
  1.000311369627848
\item
  0.999963363162325
\item
  1.000003071985154
\item
  0.999999828557529
\item
  1.000000005621326
\item
  0.999999999919745
\end{itemize}

\section{Problem 3. Application of system of linear
equations}\label{problem-3.-application-of-system-of-linear-equations}

Script:

\begin{verbatim}
% alpha value
alpha = cos(pi/4);

% create and populate matrix
A = sparse(21, 21);

A(1,1) = -alpha;
A(1,4) = 1;
A(1,5) = alpha;

A(2,1) = alpha;
A(2,3) = 1;
A(2,5) = alpha;

A(3,2) = -1;
A(3,6) = 1;

A(4,3) = -1;
A(4,18) = 1;

A(5,4) = -1;
A(5,8) = 1;

A(6,7) = 1;

A(7,5) = -alpha;
A(7,6) = -1;
A(7,9) = alpha;
A(7,10) = 1;

A(8,5) = -alpha;
A(8,7) = -1;
A(8,9) = -alpha;
A(8,19) = 1;

A(9,8) = -1;
A(9,9) = -alpha;
A(9,12) = 1;
A(9,13) = alpha;

A(10,9) = alpha;
A(10,11) = 1;
A(10,13) = alpha;

A(11,10) = -1;
A(11,14) = 1;

A(12,11) = -1;
A(12,20) = 1;

A(13,13) = -alpha;
A(13,14) = -1;
A(13,17) = 1;

A(14,13) = -alpha;
A(14,15) = -1;
A(14,21) = 1;

A(15,12) = -1;
A(15,16) = alpha;

A(16,15) = 1;
A(16,16) = alpha;

A(17,16) = alpha;
A(17,17) = 1;

% Given forces
A(18,18) = 1;
A(19,19) = 1;
A(20,20) = 1;
A(21,21) = 1;


% Loop over four situations
disp('Forces from f1-f17 then F1-F4.')

% four forces
force =[10, 15, 0, 10;
        15, 0, 0, 10;
        10, 0, 20, 0;
        0, 10, 10, 0];
for i = 1:4
    disp(strcat("Test Case: ",num2str(i)))
    forces = force(i,:);
    b = zeros(1,17);
    b = [b forces];
    b = b';
    disp(A\b)
end
\end{verbatim}

Output:

Problem3Script

Forces from f1-f17 then F1-F4.

Test Case: 1

\begin{itemize}
\tightlist
\item
  -26.870057685088803
\item
  19.000000000000000
\item
  10.000000000000000
\item
  -28.000000000000000
\item
  12.727922061357853
\item
  19.000000000000000
\item
  0
\item
  -28.000000000000000
\item
  8.485281374238570
\item
  22.000000000000000
\item
  0
\item
  -15.999999999999996
\item
  -8.485281374238570
\item
  22.000000000000000
\item
  16.000000000000000
\item
  -22.627416997969519
\item
  16.000000000000000
\item
  10.000000000000000
\item
  15.000000000000000
\item
  0
\item
  10.000000000000000
\end{itemize}

Test Case: 2

\begin{itemize}
\tightlist
\item
  -19.798989873223331
\item
  14.000000000000000
\item
  15.000000000000000
\item
  -13.000000000000000
\item
  -1.414213562373095
\item
  14.000000000000000
\item
  0
\item
  -13.000000000000000
\item
  1.414213562373095
\item
  11.999999999999998
\item
  0
\item
  -10.999999999999998
\item
  -1.414213562373095
\item
  11.999999999999998
\item
  10.999999999999998
\item
  -15.556349186104043
\item
  10.999999999999998
\item
  15.000000000000000
\item
  0
\item
  0
\item
  10.000000000000000
\end{itemize}

Test Case: 3

\begin{itemize}
\tightlist
\item
  -22.627416997969522
\item
  16.000000000000000
\item
  10.000000000000000
\item
  -22.000000000000000
\item
  8.485281374238571
\item
  16.000000000000000
\item
  0
\item
  -22.000000000000000
\item
  -8.485281374238570
\item
  28.000000000000000
\item
  20.000000000000000
\item
  -14.000000000000000
\item
  -19.798989873223331
\item
  28.000000000000000
\item
  14.000000000000000
\item
  -19.798989873223331
\item
  14.000000000000000
\item
  10.000000000000000
\item
  0
\item
  20.000000000000000
\item
  0
\end{itemize}

Test Case: 4

\begin{itemize}
\tightlist
\item
  -14.142135623730953
\item
  10.000000000000000
\item
  0
\item
  -20.000000000000000
\item
  14.142135623730951
\item
  10.000000000000000
\item
  0
\item
  -20.000000000000004
\item
  0.000000000000001
\item
  20.000000000000000
\item
  10.000000000000000
\item
  -10.000000000000000
\item
  -14.142135623730949
\item
  20.000000000000000
\item
  10.000000000000000
\item
  -14.142135623730949
\item
  10.000000000000000
\item
  0
\item
  10.000000000000000
\item
  10.000000000000000
\item
  0
\end{itemize}

The fact that I get negative forces in my output isn't necessarily a
problem, it just means that they were going the opposite direction to
which I expected them to be going.


\end{document}
