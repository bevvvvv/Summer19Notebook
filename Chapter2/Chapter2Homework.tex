\documentclass[]{article}
\usepackage{lmodern}
\usepackage{amssymb,amsmath}
\usepackage{ifxetex,ifluatex}
\usepackage{fixltx2e} % provides \textsubscript
\ifnum 0\ifxetex 1\fi\ifluatex 1\fi=0 % if pdftex
  \usepackage[T1]{fontenc}
  \usepackage[utf8]{inputenc}
\else % if luatex or xelatex
  \ifxetex
    \usepackage{mathspec}
  \else
    \usepackage{fontspec}
  \fi
  \defaultfontfeatures{Ligatures=TeX,Scale=MatchLowercase}
\fi
% use upquote if available, for straight quotes in verbatim environments
\IfFileExists{upquote.sty}{\usepackage{upquote}}{}
% use microtype if available
\IfFileExists{microtype.sty}{%
\usepackage{microtype}
\UseMicrotypeSet[protrusion]{basicmath} % disable protrusion for tt fonts
}{}
\usepackage[margin=1in]{geometry}
\usepackage{hyperref}
\hypersetup{unicode=true,
            pdftitle={Chapter 2: Polynomial Interpolation},
            pdfborder={0 0 0},
            breaklinks=true}
\urlstyle{same}  % don't use monospace font for urls
\usepackage{graphicx,grffile}
\makeatletter
\def\maxwidth{\ifdim\Gin@nat@width>\linewidth\linewidth\else\Gin@nat@width\fi}
\def\maxheight{\ifdim\Gin@nat@height>\textheight\textheight\else\Gin@nat@height\fi}
\makeatother
% Scale images if necessary, so that they will not overflow the page
% margins by default, and it is still possible to overwrite the defaults
% using explicit options in \includegraphics[width, height, ...]{}
\setkeys{Gin}{width=\maxwidth,height=\maxheight,keepaspectratio}
\IfFileExists{parskip.sty}{%
\usepackage{parskip}
}{% else
\setlength{\parindent}{0pt}
\setlength{\parskip}{6pt plus 2pt minus 1pt}
}
\setlength{\emergencystretch}{3em}  % prevent overfull lines
\providecommand{\tightlist}{%
  \setlength{\itemsep}{0pt}\setlength{\parskip}{0pt}}
\setcounter{secnumdepth}{5}
% Redefines (sub)paragraphs to behave more like sections
\ifx\paragraph\undefined\else
\let\oldparagraph\paragraph
\renewcommand{\paragraph}[1]{\oldparagraph{#1}\mbox{}}
\fi
\ifx\subparagraph\undefined\else
\let\oldsubparagraph\subparagraph
\renewcommand{\subparagraph}[1]{\oldsubparagraph{#1}\mbox{}}
\fi

%%% Use protect on footnotes to avoid problems with footnotes in titles
\let\rmarkdownfootnote\footnote%
\def\footnote{\protect\rmarkdownfootnote}

%%% Change title format to be more compact
\usepackage{titling}

% Create subtitle command for use in maketitle
\newcommand{\subtitle}[1]{
  \posttitle{
    \begin{center}\large#1\end{center}
    }
}

\setlength{\droptitle}{-2em}

  \title{Chapter 2: Polynomial Interpolation}
    \pretitle{\vspace{\droptitle}\centering\huge}
  \posttitle{\par}
    \author{}
    \preauthor{}\postauthor{}
    \date{}
    \predate{}\postdate{}
  

\begin{document}
\maketitle

{
\setcounter{tocdepth}{2}
\tableofcontents
}
\section{Problem 1}\label{problem-1}

Consider the polynomial interpolation for the following data points:

\[
\begin{tabular}{c|c|c|c|}
x & 1 & 2 & 3\\
y & 3 & 4 & 5
\end{tabular}
\]

\subsection{\texorpdfstring{a. Write down the linear system in matrix
form for solving the coefficients a\textsubscript{i} of the polynomial
p\textsubscript{n}(x).}{a. Write down the linear system in matrix form for solving the coefficients ai of the polynomial pn(x).}}\label{a.-write-down-the-linear-system-in-matrix-form-for-solving-the-coefficients-ai-of-the-polynomial-pnx.}

\[
\left(\begin{array}{ccc} 
1 & 1 & 1\\
4 & 2 & 1\\
9 & 3 & 1 
\end{array}\right)
\left(\begin{array}{c} 
a_2 \\
a_1 \\
a_0
\end{array}\right) =
\left(\begin{array}{c}
3 \\
4 \\
5
\end{array}\right)
\]

\subsection{b. Use the lagrange interpolation process to obtain a
polynomial to approximate these data
points.}\label{b.-use-the-lagrange-interpolation-process-to-obtain-a-polynomial-to-approximate-these-data-points.}

First let's define some cardinal functions:

\[I_0(x)=\frac{(x - x_1)(x-x_2)}{(x_0-x_1)(x_0-x_2)}=\frac{(x-2)(x-3)}{(1-2)(1-3)}=\frac{(x-2)(x-3)}{2}=\frac12(x-2)(x-3)\]
\[I_1(x)=\frac{(x - x_0)(x-x_2)}{(x_1-x_0)(x_1-x_2)}=\frac{(x-1)(x-3)}{(2-1)(2-3)}=\frac{(x-1)(x-3)}{-1}=-(x-1)(x-3)\]
\[I_2(x)=\frac{(x - x_0)(x-x_1)}{(x_2-x_0)(x_2-x_1)}=\frac{(x-1)(x-2)}{(3-1)(3-2)}=\frac{(x-1)(x-2)}{2}=\frac12(x-1)(x-2)\]

Then plug our cardinal functions into the equation:

\[P_2(x)=I_0(x)y_0+I_1(x)y_1+I_2(x)y_2 = 3*(\frac12(x-2)(x-3)) + 4*(-(x-1)(x-3)) + 5*(\frac12(x-1)(x-2))\]
\[P_2(x) = \frac32(x^2-5x+6)-4(x^2-4x+3)+\frac52(x^2-3x+2)\]
\[P_2(x) = \frac32x^2-4x^2+\frac52x^2-15x+16x+9-12+5\] \[P_2(x) = x+2\]

Which is a polynomial that does in fact contain all of our given data
points!

\section{Problem 2}\label{problem-2}

\subsection{a. Use the Lagrange interpolation process to obtain a
polynomial of least degree that assumes these
values}\label{a.-use-the-lagrange-interpolation-process-to-obtain-a-polynomial-of-least-degree-that-assumes-these-values}

\[
\begin{tabular}{c|c|c|c|c|}
x & 0 & 2 & 3 & 4\\
y & 7 & 11 & 28 & 63
\end{tabular}
\]

First let's define some cardinal functions.

\[I_0(x) = \frac{(x-x_1)(x-x_2)(x-x_3)}{(x_0-x_1)(x_0-x_2)(x_0-x_3)}=\frac{(x-2)(x-3)(x-4)}{(0-2)(0-3)(0-4)}=\frac{(x-2)(x-3)(x-4)}{-24}\]
\[I_1(x)=\frac{(x-x_0)(x-x_2)(x-x_3)}{(x_1-x_0)(x_1-x_2)(x_1-x_3)}=\frac{(x-0)(x-3)(x-4)}{(2-0)(2-3)(2-4)}=\frac{x(x-3)(x-4)}{4}\]
\[I_2(x)=\frac{(x-x_0)(x-x_1)(x-x_3)}{(x_2-x_0)(x_2-x_1)(x_2-x_3)}=\frac{(x-0)(x-2)(x-4)}{(3-0)(3-2)(3-4)}=\frac{x(x-2)(x-4)}{-3}\]
\[I_3(x)=\frac{(x-x_0)(x-x_1)(x-x_2)}{(x_3-x_0)(x_3-x_1)(x_3-x_2)}=\frac{(x-0)(x-2)(x-3)}{(4-0)(4-2)(4-3)}=\frac{x(x-2)(x-3)}{8}\]

Then plug our cardinal functions into the equation:

\[P_3(x)=I_0(x)y_0+I_1(x)y_1+I_2(x)y_2+I_3(x)y_3\]
\[P_3(x)=7*(\frac{(x-2)(x-3)(x-4)}{-24}) + 11*(\frac{x(x-3)(x-4)}{4}) + 28*(\frac{x(x-2)(x-4)}{-3}) + 63*(\frac{x(x-2)(x-3)}{8})\]
\[P_3(x)=\frac{-7}{24}(x^3-9x^2+26x-24)+\frac{11}4(x^3-7x^2+12x)+\frac{-28}3(x^3-6x^2+8x)+\frac{63}8(x^3-5x^2+6x)\]
\[P_3(x)=\frac{-7}{24}x^3+\frac{11}4x^3-\frac{28}3x^3+\frac{63}8x^3+\frac{63}{24}x^2-\frac{77}4x^2+\frac{168}3x^2-\frac{315}8x^2-\frac{182}{24}x+\frac{132}4x-\frac{224}3x+\frac{378}8x+7\]
\[P_3(x)=x^3-2x+7\]

And this polynomial includes all of our sample points!

\subsection{b. For the points in the previous table, find the Newton's
form of the interpolating polynomial. Show that the two polynomials
obtained are identical, although their forms may
differ.}\label{b.-for-the-points-in-the-previous-table-find-the-newtons-form-of-the-interpolating-polynomial.-show-that-the-two-polynomials-obtained-are-identical-although-their-forms-may-differ.}

The general equation for n = 3:

\[P_3(x) = a_0 + a_1(x-x_0) + a_2(x-x_0)(x-x_1)+a_3(x-x_0)(x-x_1)(x-x_2)\]

Then let's calculate the coefficients using our triangular matrix:

\[
\begin{tabular}{c|c|c|c|c|}
0 & 7 \\
2 & 11 & 2\\
3 & 28 & 17 & 5\\
4 & 63 & 35 & 9 & 1\\
\end{tabular}
\]

Which results in the polynomial:

\[P_3(x) = 7 + 2x + 5x(x-2)+x(x-2)(x-3)\] \[P_3(x) = x^3-2x+7\]

This is identical to the simplified polynomial in step a!

\subsection{\texorpdfstring{c. The polynomial
\(p(x) = x^4 - x^3 + x^2 - x + 1\) has the values
shown}{c. The polynomial p(x) = x\^{}4 - x\^{}3 + x\^{}2 - x + 1 has the values shown}}\label{c.-the-polynomial-px-x4---x3-x2---x-1-has-the-values-shown}

\[
\begin{tabular}{c|c|c|c|c|c|c|}
x & -2 & -1 & 0 & 1 & 2 & 3 \\
p(x) & 31 & 5 & 1 & 1 & 11 & 61
\end{tabular}
\]

Find a polynomial q(x) that takes the same values (you don't need to
expand it):

If we proceed with Lagrange form, then we can start with cardinal
functions.

\[I_0(x) = \frac{(x+1)x(x-1)(x-2)(x-3)}{(-2+1)(-2)(-2-1)(-2-2)(-2-3)}=\frac{-1}{120}x(x+1)(x-1)(x-2)(x-3)\]
\[I_1(x) = \frac{(x+2)x(x-1)(x-2)(x-3)}{(-1+2)(-1)(-1-1)(-1-2)(-1-3)}=\frac{1}{24}x(x+2)(x-1)(x-2)(x-3)\]
\[I_2(x) = \frac{(x+2)(x+1)(x-1)(x-2)(x-3)}{(0+2)(0+1)(0-1)(0-2)(0-3)}=\frac{-1}{12}(x+2)(x+1)(x-1)(x-2)(x-3)\]
\[I_3(x) = \frac{x(x+2)(x+1)(x-2)(x-3)}{(1+2)(1+1)(1)(1-2)(1-3)}=\frac{1}{12}x(x+2)(x+1)(x-2)(x-3)\]
\[I_4(x) = \frac{x(x+2)(x+1)(x-1)(x-3)}{(2+2)(2+1)(2)(2-1)(2-3)}=\frac{-1}{24}x(x+2)(x+1)(x-1)(x-3)\]
\[I_5(x) = \frac{x(x+2)(x+1)(x-1)(x-2)}{(3+2)(3+1)(3)(3-1)(3-2)}=\frac{1}{120}x(x+2)(x+1)(x-1)(x-2)\]

Then plug them into our polynomial:

\[P_5(x) = 31*(\frac{-1}{120}x(x+1)(x-1)(x-2)(x-3)) + 5*(\frac{1}{24}x(x+2)(x-1)(x-2)(x-3)) ...\]
\[+ \frac{-1}{12}(x+2)(x+1)(x-1)(x-2)(x-3) + \frac{1}{12}x(x+2)(x+1)(x-2)(x-3) + 11*(\frac{-1}{24}x(x+2)(x+1)(x-1)(x-3)) ...\]
\[+ 61*(\frac{1}{120}x(x+2)(x+1)(x-1)(x-2))\]


\end{document}
