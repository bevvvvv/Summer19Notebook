\documentclass[]{article}
\usepackage{lmodern}
\usepackage{amssymb,amsmath}
\usepackage{ifxetex,ifluatex}
\usepackage{fixltx2e} % provides \textsubscript
\ifnum 0\ifxetex 1\fi\ifluatex 1\fi=0 % if pdftex
  \usepackage[T1]{fontenc}
  \usepackage[utf8]{inputenc}
\else % if luatex or xelatex
  \ifxetex
    \usepackage{mathspec}
  \else
    \usepackage{fontspec}
  \fi
  \defaultfontfeatures{Ligatures=TeX,Scale=MatchLowercase}
\fi
% use upquote if available, for straight quotes in verbatim environments
\IfFileExists{upquote.sty}{\usepackage{upquote}}{}
% use microtype if available
\IfFileExists{microtype.sty}{%
\usepackage{microtype}
\UseMicrotypeSet[protrusion]{basicmath} % disable protrusion for tt fonts
}{}
\usepackage[margin=1in]{geometry}
\usepackage{hyperref}
\hypersetup{unicode=true,
            pdftitle={Chapter 3: Splines},
            pdfborder={0 0 0},
            breaklinks=true}
\urlstyle{same}  % don't use monospace font for urls
\usepackage{graphicx,grffile}
\makeatletter
\def\maxwidth{\ifdim\Gin@nat@width>\linewidth\linewidth\else\Gin@nat@width\fi}
\def\maxheight{\ifdim\Gin@nat@height>\textheight\textheight\else\Gin@nat@height\fi}
\makeatother
% Scale images if necessary, so that they will not overflow the page
% margins by default, and it is still possible to overwrite the defaults
% using explicit options in \includegraphics[width, height, ...]{}
\setkeys{Gin}{width=\maxwidth,height=\maxheight,keepaspectratio}
\IfFileExists{parskip.sty}{%
\usepackage{parskip}
}{% else
\setlength{\parindent}{0pt}
\setlength{\parskip}{6pt plus 2pt minus 1pt}
}
\setlength{\emergencystretch}{3em}  % prevent overfull lines
\providecommand{\tightlist}{%
  \setlength{\itemsep}{0pt}\setlength{\parskip}{0pt}}
\setcounter{secnumdepth}{5}
% Redefines (sub)paragraphs to behave more like sections
\ifx\paragraph\undefined\else
\let\oldparagraph\paragraph
\renewcommand{\paragraph}[1]{\oldparagraph{#1}\mbox{}}
\fi
\ifx\subparagraph\undefined\else
\let\oldsubparagraph\subparagraph
\renewcommand{\subparagraph}[1]{\oldsubparagraph{#1}\mbox{}}
\fi

%%% Use protect on footnotes to avoid problems with footnotes in titles
\let\rmarkdownfootnote\footnote%
\def\footnote{\protect\rmarkdownfootnote}

%%% Change title format to be more compact
\usepackage{titling}

% Create subtitle command for use in maketitle
\newcommand{\subtitle}[1]{
  \posttitle{
    \begin{center}\large#1\end{center}
    }
}

\setlength{\droptitle}{-2em}

  \title{Chapter 3: Splines}
    \pretitle{\vspace{\droptitle}\centering\huge}
  \posttitle{\par}
    \author{}
    \preauthor{}\postauthor{}
    \date{}
    \predate{}\postdate{}
  

\begin{document}
\maketitle

{
\setcounter{tocdepth}{2}
\tableofcontents
}
\section{Problem 1}\label{problem-1}

\subsection{a. Determine whether the function is a linear
spline}\label{a.-determine-whether-the-function-is-a-linear-spline}

Let's check the properties. First of all it does have a degree of 1 or
less on each piece of the polynomial, so that property passes. Next
let's check for continuity on the inner knots.

At x = 0.5, \(S_0(0.5) = 0.5\) and \(S_1(0.5) = 0.5 = 2 * 0 = 0.5\).
Continuous here.

At x = 2, \(S_1(2) = 0.5 + 2 * 1.5 = 3.5\) and \(S_2(2)=2 + 1.5 = 3.5\).
Continuous here.

This function S(x) must then be a linear spline.

\subsection{b. Do there exist a,b,c,d so the function is a natural cubic
spline?}\label{b.-do-there-exist-abcd-so-the-function-is-a-natural-cubic-spline}

Recall that a natural cubic spline uses the condition that the second
derivative at the end knots both equal to zero.

Let's set up our system:

\[S_0''(-1) = 6ax + 2 = 0\] \[-6a = -2\] \[a = \frac23\]

\[S_1''(1) = 6bx + 2 = 0\] \[6b = -2\] \[b = -\frac23\]

So yes, there do exist an a,b,c, and d so the function is a natural
cubic spline. If a = 2/3 and b = -2/3, then the two end points are
points of inflection.

\subsection{c. Determine whether f is a cubic spline with knots -1, 0,
1, and
2}\label{c.-determine-whether-f-is-a-cubic-spline-with-knots--1-0-1-and-2}

First of all we can confirm that each piece of the function is degree 3
or less. Second let's test for continuity between knots.

At 0:

\[S_0(0) = S_1(0)\] \[1 + 2(0+1) + (0+1)^3=3+5*0+3*0\] \[1 + 2 + 1 = 3\]
\[3 = 3\]

That works, next at 1:

\[S_1(1) = S_2(1)\] \[3 + 5 + 3 = 11 + 0 + 3 * 0 + 0\] \[11 = 11\]

That works, check first derivative at 0:

\[S_0'(0) = S_1'(0)\] \[2+3(0+1)^2=5 + 6 * 0\] \[2 + 3 = 5\] \[5 = 5\]

That works, next at 1:

\[S_1'(1) = S_2'(1)\] \[5 + 6 * 1 = 6(1-1)+3(1-1)^2+1\] \[5 \ne 1\]

Therefore the function f is NOT a cubic spline, since it is not (k-1)
times differntiable at it's inner knots.

\subsection{d. Determine the values of a,b,c such that the function is a
linear
spline.}\label{d.-determine-the-values-of-abc-such-that-the-function-is-a-linear-spline.}

Let's create our system of equations (for three unknowns).

\[S_0(-1) = s_1(-1)\] \[S_1(0) = S_2(0)\] \[S_2(1) = s_3(1)\]

Now let's solve one at a time from the end.

\[c(1) + 3(1-1) = 4\] \[c = 4\]

Now that we know c is four let's move to the second equation.

\[a(0 + 1) + b * 0 = 4 * 0 + 3(0-1)\] \[a = -3\]

Then move on to our first equation we wrote.

\[-3(-1 + 1) + b * -1 = -1 + 1\] \[-b = 0\] \[b = 0\]

So we can conclude taht a = -3, b = 0, and c = 4 gives us a linear
spline.

\subsection{e. Determine the values of a,b,c such that the function is a
linear
spline.}\label{e.-determine-the-values-of-abc-such-that-the-function-is-a-linear-spline.}

Let's create our system of equations (for three unknowns).

\[S_0(-1) = s_1(-1)\] \[S_1(0) = S_2(0)\] \[S_2(1) = s_3(1)\]

Now let's solve one at a time from the the end.

\[c(1) + 2(1-1) = 5\] \[c = 5\]

Now we know c is five let's move to the second equation.

\[a(0+1) + b * 0 = 5*0 + 2(0-1)\] \[a=-2\]

Then move on to our first equation we wrote.

\[-1 + 3 = -2(-1 + 1) + b * -1\] \[2 = -b\] \[b=-2\]

So we can conclude that a = b = -2, and c = 5 gives us a linear spline.

\section{Problem 2}\label{problem-2}

Given a set of data

\[
\begin{tabular}{c|c|c|c|c|c}
t$_i$ & 1.2 & 1.5 & 1.6 & 2.0 & 2.2\\
y$_i$ & 0.4275 & 1.139 & 0.8736 & -0.9751 & -0.1536
\end{tabular}
\]

\subsection{a.}\label{a.}

Let L(x) be the linear spline that interpolates the data. Describe what
L(x) consists of, and what conditions it has to satisfy. Find L(x), and
compute the value for L(1.8).

There are two main conditions a linear spline would have to fulfill. It
would have to be of degree 1 polynomial on each piece, and L(x) would
have to be continous at each knot denoted in the table.

For a linear spline we can just use the equation of a linear point as
written:

\[S_i(x) = y_i + \frac{y_{i+1}-y_i}{t_{i+1}-t_i}(x-t_i)\]

For i = 0:

\[S_0(x) = 0.4275 + \frac{1.139 - 0.4275}{1.5-1.2}(x-1.2)\]
\[S_0(x) = 0.4275 + 2.3717(x-1.2)\]

For i = 1

\[S_1(x) = 1.139 + \frac{0.8736 - 1.139}{1.6-1.5}(x-1.5)\]
\[S_1(x) = 1.139 - 2.654(x-1.5)\]

For i = 2

\[S_2(x) = 0.8736 + \frac{-0.9751 - 0.8736}{2.0-1.6}(x-1.6)\]
\[S_2(x) = 0.8736 - 4.6218(x-1.6)\]

For i = 3

\[S_3(x) = -0.9751 + \frac{-0.1536 + 0.9751}{2.2-2.0}(x-2.0)\]
\[S_3(x) = -0.9751 + 4.1075(x-2.0)\]

\[
S(x) = \left\{
        \begin{array}{ll}
            0.4275 + 2.3717(x-1.2) & \quad 1.2 \leq x \leq 1.5 \\
            1.139 - 2.654(x-1.5) & \quad 1.5 \leq x \leq 1.6 \\
            0.8736 - 4.6218(x-1.6) & \quad 1.6 \leq x \leq 2.0\\
            -0.9751 + 4.1075(x-2.0) & \quad 2.0 \leq x \leq 2.2
        \end{array}
    \right.
\]

\[S(1.8) = S_2(1.8) = 0.8736 - 4.6218(1.8 - 1.6) =  -0.05076\]


\end{document}
